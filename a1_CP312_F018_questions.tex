\documentclass[12pt]{article}
\addtolength{\topmargin}{-1.5cm}
\addtolength{\textheight}{3.5cm}
\addtolength{\oddsidemargin}{-2cm}
\addtolength{\evensidemargin}{-2cm}
\addtolength{\textwidth}{3.4cm}
\begin{document}

{\Large \bf
\begin{center}

CP 312, Fall 2018 \\

Assignment 1 \ \ \ (5\% of the final grade)\\

\end{center}
} \vspace{0.2cm}
There are five questions on this assignment. Note
that all logarithms in this assignment are base 2, i.e. $\log f(n)
\equiv \log_2 f(n)$.

\vspace{0.2cm}

\begin{enumerate}

% question 1
\item{[2 marks]} Using the definition of $\Theta$-notation,
prove that $(13n+3) (9n + 1) (\log (4n^2+100))  \in \Theta(n^2 \log n)$.

\vspace{0.5cm}

% question 2
\item{[10 marks]} For each of the following pairs of functions $f(n)$
and $g(n)$, determine the {\bf most appropriate symbol} in the set $\{ O,
o, \Theta, \Omega, \omega \}$ to complete the statement \\
$f(n) \in $
\verb| __ | $\left( g(n) \right)$ (if one of the symbols applies at all).\\
Remark: Although $n \in O(2^n)$ is technically correct, $n \in o(2^n)$ is
more appropriate as it gives more information. \\
Justify your answer.

You may use the following information, where $f(n) \ll g(n)$ is
shorthand for $f(n) \in o(g(n)$:

$$1 \ll \log \log n \ll \log n \ll \log^2 n \ll \sqrt{n} \ll n \ll n \log n \ll n^2 \ll 2^n \ll n!$$

Furthermore, for all positive real $a$ and $b$, $n^a \in o(n^b)$ if $ a < b$, $\log^a n \in
o(n^b)$, and $n^a \in o(2^n)$.

\vspace{0.5cm}

(a) $f(n) = 1052n^3+10n^2+10001, \ \ g(n) = \frac{2}{10000000} n^4+2n$;\\
(b) $f(n) = \log^3(n^{10}), \ \ g(n) = \sqrt{\sqrt{n}}$;\\
(c) $f(n) = n^4\log n, \ \ g(n) = 2 n^4 {\log n^{2004}}$;\\
(d) $f(n) = 16^{\log \sqrt{n}}, \ \ g(n) = n^{2}$\ ; \\
(e) $f(n) = n^{2+\sin \frac{n \pi}{2}}, \ \ g(n) = n^{5/3}$.
\vspace{0.5cm}

% question 3
\item{[10 marks]}
Consider each of the following statements, assuming that all functions are non-negative: \\
\vspace{0.5cm}

\noindent
a) if $f_1(n) \in \Theta(g(n))$ and $f_2(n) \in \Theta(g(n))$, then
$f_1(n)-f_2(n) \in O(1)$;\\
b) if $f_1(n) \in O(g_1(n))$ and $f_2(n) \in O(g_2(n))$, then
$f_1(n)f_2(n) \in \Theta(g_1(n)g_2(n))$; \\
c) if $f_1(n) \in \Theta(g(n))$ and $f_2(n) \in \Theta(g(n))$, then
$f_1(n)/f_2(n) \in \Theta(1)$;\\
d) if $\log (f(n)) \in O(\log(g(n)))$
then $f(n) \in O(g(n))$.\\

%\vspace{0.5cm}
For each statement: if the statement is true then provide a proof
that starts with the formal definition of the order notation
utilized in the statement. If the statement is false then provide a
counter example and  demonstrate why the statement is false.

%\vspace{0.5cm}

% question 4
\item{[4 marks]} Analyze the following pseudocode and give a tight
($\Theta$) bound on the running time
as a function of $n$. You can assume that all individual instructions
are elementary. Show your work. \\
\verb|     |$l$ \verb| := | 0;\\
\verb|     for |$i=1$ \verb| to | $ n $ \verb| do |\\
\verb|       for |$j=i$ \verb| to | $n \cdot i$ \verb|  do |\\
%%\verb|          for |$k=1$ \verb| to | $\lceil \log j \rceil$ \verb|  do |\\
\verb|          |$l $\verb| := | $l+2*i+3*j$\\
%%\verb|          od  |\\
\verb|       od  |\\
\verb|     od.|\\

\vspace{0.5cm}

% question 5
\item{[4 marks]} Analyze the following pseudocode and give a tight
($\Theta$) bound on the running time
as a function of $n$. You can assume that all individual instructions
are elementary. Show your work. \\
\\
\verb|     |$k$ \verb| := | $n$; $s$ \verb| := | $0$;\\
%%\verb|     while |$k<n$ \verb| do |\\
\verb|     for |$i=1$ \verb| to | $k$ \verb|  do |\\
\verb|        for |$j=1$ \verb| to | $\left\lceil \log i \right\rceil$ \verb|  do |\\
\verb|            |$s $\verb| := | $s+i*j$\\
\verb|        od  |\\
\verb|     od  |\\
\verb|     for |$i=11k+1$ \verb| to | $33k$ \verb|  do |\\
\verb|        |$s $\verb| := | $s+1$\\
\verb|     od  |\\
%%\verb|       |$k$ \verb| := | $2$ \verb|*| $k$\\
%%\verb|     od.|\\


\end{enumerate}


\end{document}
